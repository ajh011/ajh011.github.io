%--beamer--------------------------------------------------------------------------------------
%	PACKAGES AND THEMES
%----------------------------------------------------------------------------------------
\documentclass[xcolor=dvipsnames]{beamer}
\usetheme{default}
\usepackage[backend=bibtex,
defernumbers=true,
style=numeric,
citestyle=ieee
]{biblatex}
%\addbibresource{ref.bib}
\usepackage{hyperref}
\usepackage{graphicx} % Allows including images
\usepackage{booktabs} % Allows the use of \toprule, \midrule and \bottomrule in tables
\usepackage[utf8]{inputenc}
\usepackage[T1]{fontenc}
\usepackage{amsmath,amssymb}
\usepackage{multirow}
\usepackage{array}

% Uncomment here for a Turkish presentation.
%\AtBeginDocument{%
%	\renewcommand\tablename{Tablo}
%}
%\AtBeginDocument{%
%	\renewcommand\figurename{\u015eekil}
%}

%----------------------------------------------------------------------------------------
%	TITLE PAGE
%----------------------------------------------------------------------------------------

% The titlo
\title[Title]
{
 Quantum Algorithims \& Qiskit	
}

\subtitle{}
\author{ Alexander J. Heilman }
 
% can specify the institute here
%\institute[]{\inst{1}Bilgisayar M�hendisli\u011fi B�l�m�\\Gebze Teknik �niversitesi\and\inst{2}Bili\u015fim Teknolojileri Enstit�s�\\Gebze Teknik �niversitesi}

\date{\today} % Date, can be changed to a custom date


%----------------------------------------------------------------------------------------
%	PRESENTATION SLIDES
%----------------------------------------------------------------------------------------

\begin{document}

\begin{frame}
    % Print the title page as the first slide
    \titlepage
\end{frame}


% Uncomment here if you want a table of contents. You should use \section etc. (explained below)
% \begin{frame}{Preface}
%     % Throughout your presentation, if you choose to use \section{} and \subsection{} commands, these will automatically be printed on this slide as an overview of your presentation
%     \tableofcontents
% \end{frame}


%------------------------------------------------

\begin{frame}{The Quantum Fourier Transform}
    The quantum Fourier transform (QFT) is a quantum implementation of the discrete
    Fourier transform

    \medskip

    \pause

    This means if you give the QFT some state $\sum_{n}^{} \vert nk+l \rangle $, the
    QFT will give back $\vert k \rangle $.

    \medskip

    \pause

    But what does that look like?
\end{frame}

%------------------------------------------------

\begin{frame}{QFT: The gory details}
    The explict action of the n-qubit QFT on some given basis vector is 
    \[\hspace{-7cm}
        \vert j_1,...,j_n \rangle  \longrightarrow
    \]

    \vspace{-.9cm}

    \[
        \frac{\left( \vert 0 \rangle  + e^{2\pi i0.j_n} \vert 1 \rangle  \right)\otimes \left( \vert 0 \rangle  + e^{2\pi i0.j_{n-1}j_n} \vert 1 \rangle \right)\otimes ...\otimes \left( \vert 0 \rangle  + e^{2\pi i0.j_1...j_n}  \vert 1 \rangle  \right)  }{2^{n/2} }
    \]. \pause
    Show circuir here

    Sorry for asking? \pause Don't be it's easier to see in matrix form
    
    
\end{frame}

%------------------------------------------------

\begin{frame}{QFT: The gory details II}
 For example with $n=3$ the QFT is 

    
    \[
      \frac{1}{2\sqrt{2}} 
    \begin{bmatrix}
    1 & 1 & 1 & 1 & 1 & 1 & 1 & 1\\
    1 & \omega &\omega ^2      &\omega ^3  &\omega ^{4} &\omega ^{5} &\omega ^{6} &\omega ^{7}\\
    1 & \omega ^2&\omega ^{4}  &\omega^{6} &1       &\omega ^2&\omega ^{4} &\omega ^{6}\\
    1 &\omega ^{3}&\omega ^{6} &\omega     &\omega ^{4} &\omega ^{7} &\omega ^2&\omega ^{5} \\
    1 &\omega ^{4}&           1&\omega^{4} &1&\omega ^{4} &1 & \omega ^{4}   \\
    1 &\omega ^{5}&\omega ^{2} &\omega^{7} &\omega ^{4} &\omega &\omega ^6&\omega ^{3} \\
    1 &\omega ^{6}&\omega ^{4} &\omega^2  &1&\omega ^{6} &\omega ^4&\omega ^{2} \\
    1 &\omega ^{7}&\omega ^{6} &\omega^{5}  &\omega ^{4} &\omega ^{3} &\omega ^2&\omega  \\
\end{bmatrix}
    \].\pause

    Not so interesting yet, but it's usefulness will soon be found in phase estimation
    
    
\end{frame}

%------------------------------------------------

\begin{frame}{Phase Estimation}
	Phase Estimation takes some given unitary operator and an associated eigenvector
    and returns the corresponding eigenvalue.\pause

    \medskip

    Phase estimation really returns $\vert k \rangle $, where the eigenvalue is $\lambda = e^{i\theta }$
    with $\theta =k \frac{2\pi }{2^{n} } $.\pause

    \medskip

    So how do we do this? \pause Easy!
\end{frame}

%------------------------------------------------

%------------------------------------------------

\begin{frame}{Phase Estimation: Not so bad!}
    show circuit here with annotated state vector evolution, compare resulting
    statevector to QFT output
\end{frame}

%------------------------------------------------

%------------------------------------------------

\begin{frame}{Amplitude Estimation}
    Amplitude estimation takes some superposition of states partitioned into a 
    good and a bad subspace, and returns a statevector nudged towards the good subspace.\pause

    \medskip

    This is accomplished with consecutive applications of a special operator 
     Show diagram here
\end{frame}

%------------------------------------------------

%------------------------------------------------

\begin{frame}{Amplitude Estimation: Working principles}
    The initial statevector can be decomposed as
    $\vert \psi  \rangle = sin(\theta )\vert \mathcal{G} \rangle +cos(\theta ) \vert \mathcal{B} \rangle  $.\pause

    \medskip

    We then construct the operator: $\mathcal{S}=\mathbb{I}-2\vert \mathcal{B} \rangle\langle \mathcal{B} \vert  $
\end{frame}

%------------------------------------------------
\begin{frame}{Qiskit}
	\begin{columns}[c] % The "c" option specifies centered vertical alignment while the "t" option is used for top vertical alignment
		

		
		\column{.5\textwidth} % Right column and width
	    Qiskit is a software devlopment kit with a python front-end partially
        developed by IBM.\pause

		\column{.45\textwidth} % Left column and width
		\textbf{Features}
		\begin{enumerate}
			\item Simulate + visualize quantum circuits you create 
                yourself\pause
			\item Compatible with IBM's current quantum computers\pause
			\item Vibrant and active online community
		\end{enumerate}
	\end{columns}
\end{frame}

%------------------------------------------------


%------------------------------------------------

\begin{frame}{Running the QFT}
    Let's run the 3-qubit QFT on some states\pause


\end{frame}


%------------------------------------------------

%------------------------------------------------

\begin{frame}{Running the QFT}
    Let's run the 3-qubit QFT on some states\pause

    \medskip

    Put bloch spheres of QFT inputs + ouputs here

\end{frame}


%------------------------------------------------

%------------------------------------------------

\begin{frame}{Finding Phases}
    Let's run the 3-qubit phase estimation algorithim on some gates\pause

    \medskip

    If we use controlled phase gates, we already know that $\vert 0 \rangle $ is an
    eigenvector.
    Put counts screenshots here

\end{frame}


%------------------------------------------------

%------------------------------------------------

\begin{frame}{Finding Phases}
    Let's run the 3-qubit phase estimation algorithim on some gates\pause

    \medskip

    If we use controlled phase gates, we already know that $\vert 0 \rangle $ is an
    eigenvector.
    Put counts screenshots here

\end{frame}


%------------------------------------------------


%------------------------------------------------

\begin{frame}[allowframebreaks]{References}
    % This might take more than one page
    %\printbibliography
\end{frame}

%------------------------------------------------

\begin{frame}
    \Huge{\centerline{Thanks}}
\end{frame}

%------------------------------------------------

	

\end{document}
