\documentclass[11pt]{beamer}
\usetheme{Goettingen}
\usepackage[utf8]{inputenc}
\usepackage{amsmath}
\usepackage{amsfonts}
\usepackage{amssymb}
\usepackage{graphicx}
\author{Alex Heilman}
\title{NV-Centers in Diamond as a Qubit Platform}
\subtitle{An Overview}
%\setbeamercovered{transparent} 
%\setbeamertemplate{navigation symbols}{} 
%\logo{} 
%\institute{} 
%\date{} 
%\subject{} 
\begin{document}

\begin{frame}
\titlepage
\end{frame}

%\begin{frame}
%\tableofcontents
%\end{frame}

\section{Qubit Criteria}
\begin{frame}{Criteria for Qubit Platforms}
Initializable

Readable

Robust

Scalable

See: DiVicenzio's Criteria
\end{frame}

\section{Defects in Diamond}
\begin{frame}{Why Diamond?}
Diamond is an ideal candidate due to low density of phonon modes (relatively high Debye temperature)

This leaves spins of defects and electrons less influenced by phonon modes, increasing their coherence times
\end{frame}

\begin{frame}{What's an NV-Center}
NV-Center refers to a type of defect in diamond lattices with several charge states (-, +, neutral). The most commonly discussed charge state is the (-) NV-Center. If unspecified, this is most likely the defect under consideration.

The basis for the NV-Centers is a Nitrogen substitution adjoined by a vacancy (Nitrogen-Vacancy) which results in a 'center' between the two with localized energy/spin  states.

These energy states may be used as a Qubit platform.
\end{frame}

\section{NV as Qubit}
\begin{frame}
The localized spin states of the defect may be used as a two-level system 
\end{frame}

\end{document}