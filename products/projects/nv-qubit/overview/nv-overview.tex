\documentclass[10pt,a4paper]{article}

\usepackage[utf8]{inputenc}
\usepackage{amsmath}
\usepackage{amsfonts}
\usepackage{amssymb}
\usepackage{fullpage}
\usepackage{svg}
\usepackage{titling}
\usepackage{tikz}
\usepackage{physics}
\usepackage{booktabs}
\usepackage{xcolor}
\usepackage{multicol}

\title{NV-Center in Diamond as a Qubit Platform}
\author{Alexander Heilman}

\setlength{\droptitle}{-8em}   % This is your set screw
%\setlength{\parindent}{0pt}
\begin{document}

\vspace{-3cm}
 
\maketitle

\begin{multicols}{2}

\section{Criteria for Qubit Platforms}
Divicenzio's criteria

Universality of gates (Paulis and CNOT).

\section{NV-Center in Diamond}
Most likely referring to (-) NV-Center without detailed reference (this is the most researched). 

\subsection{Spin States}
Center's spin state can be in singlet or triplet state.

\section{NV-Center as a Qubit Platform}

\subsection{Gates/Control}
Light pulses


\section{Current State of Experiment (2023)}
Recently, 10-qubit control was performed using the center's spin state and the surrounding nuclear spins!

Spin-photon entanglement has been demonstrated!

\section{Conclusion \& Outlook}

\section{Resources/Literature Review}
For a good introduction and (somewhat outdated) overview of the experimental progress in the technique, see
\cite{childress2013diamond}.

For an in-depth analysis of all defect systems in diamond, see this thesis %\cite{•}.
\end{multicols}

\bibliographystyle{plain}

\bibliography{nv} 

\end{document}
