\documentclass[11pt]{beamer}
\usetheme{Goettingen}
\usepackage[utf8]{inputenc}
\usepackage{amsmath}
\usepackage{amsfonts}
\usepackage{amssymb}
\usepackage{graphicx}
\usepackage{hyperref}
\usepackage{extarrows}
\usepackage[hang,flushmargin]{footmisc}

\title{Point Group Equivariant Convolutional Graph Neural Networks}
\author{Alex Heilman}
%\setbeamercovered{transparent} 
%\setbeamertemplate{navigation symbols}{} 
%\logo{} 
%\date{} 
\addtobeamertemplate{navigation symbols}{}{%
    \usebeamerfont{footline}%
    \usebeamercolor[fg]{footline}%
    \hspace{1em}%
    \insertframenumber/\inserttotalframenumber
}


\usepackage[style=numeric,backend=bibtex]{biblatex}
\addbibresource{chgcnn.bib}

\newenvironment{boxed2}
    {\begin{center}
    \begin{tabular}{|p{0.95\textwidth}|}
    \hline\\
    }
    { 
    \\\\\hline
    \end{tabular} 
    \end{center}
    }


\renewbibmacro*{\tiny cite:title}{\tiny%
  \printtext[bibhyperref]{%
    \printfield[citetitle]{labeltitle}%
    \setunit{\space}%
    \printtext[parens]{\printdate}%
  }%
}

\begin{document}

\begin{frame}
\titlepage
\end{frame}

%\begin{frame}
%\tableofcontents
%\end{frame}


\begin{frame}{Group Definition}
	A group $G$ is a set of elements $\lbrace g_1, ..., g_n\rbrace$ with a binary operation $*:G\times G \rightarrow G$ between elements that satisfies the conditions of identity, associativity, invertability, and closure.
	
	\begin{boxed2}
		
		\vspace{-.5cm}
		
		\textbf{Example: General Linear Group}
		
		The general linear group $GL(V)$ formed over some vector space $V$ is the set of non-singular $d_v\times d_v$ matrices acting on $V$ with the group operation being matrix multiplication. The general linear group is itself a vector space.
	\end{boxed2}
\end{frame}

\begin{frame}{Vector Space Definition}
	A vector space $V$ over a field $K$ is a group of vectors equipped with a distributive scalar multiplication. Vectors are often defined by way of a basis set that spans the space under scalar multiplication.
	

	\begin{boxed2}
		
		\vspace{-.61cm}
		
		\textbf{Example: $\mathbb{R}^3$, Real 3 Dimensional Space} 
		
		Locations in physical space may be modeled with a three dimensional vector space  $\mathbb{R}^3$ over the real numbers $\mathbb{R}$ with basis functions $\hat{x},\hat{y},\hat{z}$.
		
		\vspace{-.3cm}
		
	\end{boxed2}	
	
	
		\begin{boxed2}
		
		\vspace{-.61cm}
		
		\textbf{Example: Functions on Real 3 Dimensional Space} 
		
		Scalar functions on physical space also form a vector space over the real numbers, albeit infinite-dimensional. In this case, the group operation between vectors (functions) is point-wise addition.
		
		\vspace{-.3cm}
		
	\end{boxed2}	
	
We may construct new vector spaces from sets of existing vector spaces by taking tensor products and direct sums.

\end{frame}



\end{document}