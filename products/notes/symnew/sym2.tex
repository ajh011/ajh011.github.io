%%%%quick build setup user setting for  knitr + pytex
%%  here->    r -e "knitr::knit('%.Rnw')"|pdflatex -synctex=1 -interaction=nonstopmode %.tex|pythontex %.pytexcode|pdflatex -synctex=1 -interaction=nonstopmode %.tex|"C:/Program Files/Adobe/Reader 11.0/Reader/AcroRd32.exe" %.pdf

%%%%quick build setup user setting for only knitr
%%  here->    r -e "knitr::knit('%.Rnw')"|pdflatex -synctex=1 -interaction=nonstopmode %.tex|"C:/Program Files/Adobe/Reader 11.0/Reader/AcroRd32.exe" %.pdf

%%%%quick build setup user setting for only pytex
%%  here->    pdflatex -synctex=1 -interaction=nonstopmode %.tex|pythontex %.pytexcode|pdflatex -synctex=1 -interaction=nonstopmode %.tex|"C:/Program Files/Adobe/Reader 11.0/Reader/AcroRd32.exe" %.pdf



\documentclass{article}





%%%%%%%%%%%%%%%%%%%%%%%%%%%%%%%%%%%%%%%%%
%ONLY DIFFERENCE FROM HDR IS NO DOC CLASS INCLUDED%
%%%%%%%%%%%%%%%%%%%%%%%%%%%%%%%%%%%%%%%%%








%  paper draft, symmetric hypergraphs, summer 2015 and summer 2016
%% revision 4/2017 for submission to JPhysA
%%%% this revision is for minor edits suggested by referees in 4/7/2017 report

%% draft style for writing process



%%%remove [titlepage] here for flush title
%\documentclass[titlepage]{article}
%\documentclass{revtex4-1}


%%% revtex aip style for journal of mathematical physics
%%% move \author and \title declarations before/after ``\begin{document}''
%\documentclass[aip]{revtex4-1}

\def\changemargin#1#2{\list{}{\rightmargin#2\leftmargin#1}\item[]}
\let\endchangemargin=\endlist 





%%%%%%%%%%%PythonTex%%%%%%%%%%%%%%%%%%%%%%%%%%%%%%%%%%%%%%%%%%%%%%%%%%%%%%%%%%%%%%%%%%%%%%%%%%%%%%%%%%%%%%%%%%%

% Engine-specific settings
% Detect pdftex/xetex/luatex, and load appropriate font packages.
% This is inspired by the approach in the iftex package.
% pdftex:

%%%%%%%%%%%%%%%%%%%%%%%%%
%if trouble remove % from next rows before % xetex:
%%%%%%%%%%%%%%%%%%%%%%%%%%%
   %\ifx\pdfmatch\undefined

\usepackage{verbatim}
\usepackage{fullpage}
\usepackage{wrapfig}
\usepackage{float}
\usepackage{transparent}
\usepackage{pstricks}


\addtolength{\oddsidemargin}{.05in}
\addtolength{\topmargin}{-.50in}
\addtolength{\textheight}{1in}
\reversemarginpar
\addtolength{\headsep}{.25in}
\parskip = 5pt plus 0pt minus 0pt
\pagestyle{myheadings}
\setlength{\parindent}{0in}



\newcommand{\pd}{\frac{\partial }{\partial \dot{q}}}
\newcommand{\Rn}{\mathbb{R}^n}
\newcommand{\M}{\mathcal{M}}

%%%%%%%%%%%%%%%%%%%%%%%%%


\title{Symplectic Notes}%%Title
\author{Alexander J. Heilman}%%Name
\date{revised: \today}%%Date
\markright{Symplectic Notes}%%Heading



%%%%%%%%%%%%%%%%%%%%%%%%%
\begin{document}

\maketitle


\pagebreak



\subsection{Problem Set 1}


1. (i) A torus is a manifold as it is locally euclidean at every point

(ii) A figure 8 is not a manifold as it intersects itself and at that point is not locally euclidean 

(iii) The manifold defined by $z-x^2-y^2=0$ requires atleast two charts in $\mathbb{R}^2$

(iv) The real projective space $\mathbb{R}P^n$ is the set of all lines passing through the origin in $\mathbb{R}^{n+1}$. Two points $\vec{x},\vec{y}\in\mathbb{R}^{n+1}$, define the same line if $\vec{x}=\alpha\vec{y}$ where $\alpha,\vec{x},\vec{y}\neq 0$. This gives us freedom to choose any point along each line as the representative for that line. 

Define $U_i$ to be the set of lines with $x_i\neq 0 \ (1\leq i\leq n+1)$. Now define charts $\phi_i:U_i\rightarrow \mathbb{R}^n$:

$$
\phi_i: (x_1,...,x_{n+1})\rightarrow \left(\frac{x_1}{x_i},...,\frac{x_{i-1}}{x_i},\frac{x_{i+1}}{x_i},...,\frac{x_{n+1}}{x_i}\right)
$$ 

For $\vec{x}\in U_i\cap U_j$, $\phi_j\circ\phi_i^{-1}=
\left(\frac{i_1 x_i}{x_j},...,\frac{x_i}{x_j} ,...,\frac{i_{j-1}x_i}{x_j},\frac{i_{j+1}x_i}{x_j},...,\frac{i_{n+1}x_i}{x_j}\right)
$

For $n+1=2$, $\vec{x}\in U_1\cap U_2$, $\phi_a\circ\phi_b^{-1}=
\left(\frac{bx_1}{x_2}\right) 
$ where $b$ is the input coordinate to the transition map and $\phi_a,\phi_b$ corresponds to $x_1,x_2$ respectively. This map is continuous as $x_1,x_2$ should never be zero according to the definition of the charts.

2. (i) Given the circle $S^1$ embedded in $\mathbb{R}^2$ via $x^2+y^2=1$ and the charts $\phi_1 ,\phi_2$:

$$
\phi_1^{-1}: (0,2\pi )\rightarrow S^1
$$

\vspace{-.3cm}

$$
\phi_1^{-1}: \theta \rightarrow (cos(\theta ),sin(\theta ))
$$

\medskip

$$
\phi_2^{-1}: (-\pi ,\pi )\rightarrow S^1
$$

\vspace{-.3cm}

$$
\phi_2^{-1}: \theta \rightarrow (cos(\theta ),sin(\theta ))
$$

It is easily seen that the charts overlap only in $(0,\pi )$, and thus $\phi_1\circ\phi_2^{-1}$ is defined in this range as $\theta_{\phi_2}\rightarrow\theta_{\phi_2}+\pi$

(ii) Given $S^2$ embedded in $\mathbb{R}^3$ via $x^2+y^2+z^2=1$, stereographic projection onto a plane under the sphere provides the chart

 $$\phi_s:S^2\rightarrow \mathbb{R}^2$$

 $$\phi_s:(x,y,z)\rightarrow (X=\frac{x}{1-z},Y=\frac{y}{1-z})
 $$
 
Which is well defined everywhere except for $z=1$, and thus atleast one more chart is required.

(iii) Polar coordinates also provide a chart $\psi:S^2\rightarrow \mathbb{R}^2$ for the sphere given by 

$$
\psi^{-1}: (\theta ,\phi)\rightarrow(x=sin\theta cos\phi\,\ y=sin\theta sin\phi, \ z=cos\theta)
$$

which also fails for one point, $\theta=0$. Composition of the inverse map of polar coordinates and stereographic projection then yields

$$
\phi_s\circ\psi^{-1} :(\theta,\phi)\rightarrow (X=\frac{sin\theta cos\phi}{1-cos\theta},Y=\frac{sin\theta sin\phi}{1-cos\theta})
$$

3. (i)Vectors are directional derivatives, scalars are real numbers. Easily seen to satisfy distributivity, associativity rules satisfied as it would be for functions.

(ii)$X[x^i(t)]$ can be interpreted as velocity in the $x^i$ direction at time $t$

\pagebreak

\subsection{Problem Set 2}

1. (i) See E-L derivation section. % Note that $\frac{\partial L}{\partial q}=\frac{\partial L}{\partial x}\frac{\partial x}{\partial t}$ think

(ii) Given $L=\frac{1}{2}g_{ij}(x_{\nu})\dot{x}^i\dot{x}^j$ where $g_{ij}$ is given to be a function of position, symmetric in $i,j$ and $g^{ik}g_{kj}=\delta^i_j$

Applying EL eq.,

$$
\frac{d}{dt}\frac{d}{d\dot{x}^{\alpha}}\left(
\frac{1}{2}g_{ij}(x_{\eta})\dot{x}^i\dot{x}^j
\right)=\frac{d}{dx^{\alpha}}\left(
\frac{1}{2}g_{ij}(x_{\eta})\dot{x}^i\dot{x}^j
\right)
$$


$$
\frac{d}{dt}\frac{d}{d\dot{x}^{\alpha}}\left(
\frac{1}{2}g_{ij}(x_{\eta})\dot{x}^i\dot{x}^j
\right)= \frac{d}{dt}\left( \frac{1}{2}g_{ij}(\delta^{\alpha}_{i}\dot{x^j}+\dot{x^i}\delta^{\alpha}_{j})\right)
=\frac{d}{dt}\left( g_{\alpha j}\dot{x^j}\right)
=g_{\alpha j}\ddot{x}^j+\frac{\partial g_{\alpha j}}{\partial x^{\eta} }\frac{\partial x^{\eta}}{\partial t}\dot{x}^{j} 
$$


\[
    =g_{\alpha j}\ddot{x}^{j}+\frac{1}{2}
\left( \frac{\partial g_{\alpha j}}{\partial x^{\eta} }\frac{\partial x^{\eta}}{\partial t}\dot{x}^{j}+
\frac{\partial g_{i \alpha }}{\partial x^{\eta} }\frac{\partial x^{\eta}}{\partial t}\dot{x}^{i}\right) 
= g_{\alpha j}\ddot{x}^{j}+\frac{1}{2}
\left( \frac{\partial g_{\alpha j}}{\partial x^{\eta} }\dot{x}^{\eta}\dot{x}^{j}+
\frac{\partial g_{i \alpha }}{\partial x^{\eta} }\dot{x}^{\eta} \dot{x}^{i}\right) 
\] 



$$
\frac{d}{dx^{\alpha}}\left(
\frac{1}{2}g_{ij}\dot{x}^i\dot{x}^j
\right)= \frac{1}{2} \frac{\partial g_{i j}}{\partial x^{\alpha } }\dot{x}^{i} \dot{x}^{j}
$$

Equating the two sides of the E-L equations then gives:

\[ 
 g_{\alpha j}\ddot{x}^{j}+\frac{1}{2}
\left( \frac{\partial g_{\alpha j}}{\partial x^{\eta} }\dot{x}^{\eta}\dot{x}^{j}+
\frac{\partial g_{i \alpha }}{\partial x^{\eta} }\dot{x}^{\eta} \dot{x}^{i}\right) 
  =  \frac{1}{2} \frac{\partial g_{i j}}{\partial x^{\alpha } }\dot{x}^{i} \dot{x}^{j}
\]


\[
    g_{\alpha j} \ddot{x}^{j} = - \frac{1}{2} \left( \frac{\partial g_{\alpha  j}}{\partial x^{\eta} }
    + \frac{\partial g_{i \alpha }}{\partial x^{\eta} } -\frac{\partial g_{ij}}{\partial x^{\alpha} } \right)
    \dot{x}^{i}\dot{x}^{j}  
\]. 




2. (i.) Given $n$ different 1-forms $\omega_i$ and a permutation $\pi$ of $\lbrace 1,...,n\rbrace$, show that 

$$
\pi(\omega_1\wedge ... \wedge \omega_n)= sgn(\pi)\omega_1\wedge...\wedge\omega_n
$$

Given the wedge product is antisymmetric and the fact that permutations are partitioned by requiring an odd or an even number of transpositions, it follows that the required rearrangement to achieve $\pi$ is equal to $(-1)^{0\ if\ even \ or\ 1\ if \ odd}$.



(ii.) Prove that the dimension of the vector space $\bigwedge^k V$ is 
$\begin{bmatrix}
n \\
k
\end{bmatrix}
$.
For a $k$-form in $n$ dimensions, there are $\frac{n!}{(n-k)!}$ choices of orders of basis vectors but due to anti-symmetry,   
there are $k!$ rearrangements of any $k$-form up to some negative phase and $\frac{n!}{(n-k)!k!}=nCk$. As such, the dimension of $\bigwedge^n V=1$.

(iii.) 3-forms in the space of $\R^3$ take the form:

$$
f(x,y,z)dx\wedge dy\wedge dz
$$

This manifold should be orientable? Jacobian always positive/unital?

3. (i.) on arch

(ii.) 1-forms on $\R$ will have the form $\omega=f(x)dx$ which is clearly closed and exact as some function can be defined $\frac{dg}{dx}=f,\ g=\int f(x) dx$.

(iii.) Given the following form in $\R^2$, 

$$
\omega=\frac{-y}{x^2+y^2}dx + \frac{x}{x^2+y^2}dy
$$

calculate $d\omega$.

$$
d\omega=\left(\frac{d}{dy}\frac{y}{x^2+y^2}+\frac{d}{dx}\frac{x}{x^2+y^2}\right) dx\wedge dy=\left(\frac{1}{x^2+y^2}+\frac{-y}{(x^2+y^2)^2}(2y)+\frac{1}{x^2+y^2}+\frac{-x}{(x^2+y^2)^2}(2x)\right) dx\wedge dy
$$

$$
=0
$$

We know $\omega\neq df$ for any function $f(x,y)$ and thus is not exact as this would entail 

$$
\frac{\partial f}{\partial x}=\frac{-y}{x^2+y^2},\ \frac{\partial f}{\partial y}=\frac{x}{x^2+y^2}
$$

but for a function we should have $\frac{\partial }{\partial y}\frac{\partial f}{\partial x}=\frac{\partial }{\partial x}\frac{\partial f}{\partial y}$, which is clearly not satisfied here as

$$
\frac{\partial }{\partial y}\frac{\partial f}{\partial x}=\frac{\partial }{\partial y}\frac{-y}{x^2+y^2}=\frac{1}{x^2+y^2}+\frac{-y}{(x^2+y^2)^2}(2y)
$$

$$
\frac{\partial }{\partial x}\frac{\partial f}{\partial y}=\frac{\partial }{\partial x}\frac{x}{x^2+y^2}=\frac{1}{x^2+y^2}+\frac{x}{(x^2+y^2)^2}(2x)
$$




\subsection{Problem Set 3}


1. (i) show that this atlas for the Mobius strip  is not orientable: charts $(x^1,x^2)$ and $(y^1,y^2)$ that overlap in regions $A_1$ of chart $x$ and $A_2$ of chart $y$ with transition map $x^1=y^1+7$, $x^2=y^2$ and overlap in regions $B_1$ of chart $x$ and $B_2$ of chart $y$ with transition map $x^1=y^1-7$, $x^2=-y^2$.

(ii) Explain why all one-dimensional manifolds are orientable: A one dimensional manifold will always have charts of the form $x^i$ where there is only one parameter $x$ for every chart $i$. Thus $J=\frac{\partial x^i}{\partial x^j}$





lecture 4 1:05:00 for ex statement

lecture 5 1:09:00

lecure 5 1:23:00

%\cite{arnold1989}

%\cite{coopersmith2017}

\end{document}
