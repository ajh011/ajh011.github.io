\documentclass[10pt,a4paper]{article}

\usepackage[utf8]{inputenc}
\usepackage{amsmath}
\usepackage{amsfonts}
\usepackage{amssymb}
\usepackage{fullpage}
\usepackage{svg}
\usepackage{titling}
\usepackage{tikz}
\usepackage{physics}
\usepackage{booktabs}
\usepackage{xcolor}
\usepackage{multicol}

\title{BCS Gap Temperature Dependence \& Landau-Ginzberg Fluctuations}
\author{Alexander Heilman}

\setlength{\droptitle}{-8em}   % This is your set screw
%\setlength{\parindent}{0pt}
\begin{document}

\vspace{-3cm}
 
\maketitle

\begin{multicols}{2}
\section{BCS Gap Temperature Dependence}
The BCS gap equation is given as the following:
$$
\frac{1}{g\nu}=\int_0^{\omega_D}d\xi\ \frac{\tanh(\lambda(\xi)/2T)}{\lambda_(\xi)}
$$
where $\lambda(T)=\sqrt{\xi^2+\Delta^2}$. 

We wish to solve for the critical temperature $T_c$, where the superconducting state is reached. We now make the assumption that at the critical temperature, $T_c$, the order parameter $\Delta$ is zero, so that $\Delta(T_c)=0$ and $\lambda(T)=|\xi |$. Further noting that the largest contribution to the integral is the region where $\tanh(x)\approx 1$, we first take the simple route and approximate the numerator as $1$, as below (making the transformation $x=\xi/2T_c$).
\begin{align*}
\frac{1}{g\nu}&\approx \int_0^{\omega_D /2T_c} \frac{1}{x}\ dx
 = \ln(\omega_D/2T_c)
\end{align*}
This gives us an approximate critical temperature of the form below:
$$
T_c = C \times \omega_D e^{-\frac{1}{g\nu}}
$$
Note that this is the same form for the critical temperature reached from consideration of the four-point correlation function of the effective BCS Hamiltonian!

Now we wish to extract some more information about the analytic behaviour of this function around the critical temperature $T_c$. The form of the integral cannot easily be treated in an analytic manner. However, we may seek a perturbative expansion of the function about the critical temperature.

To achieve this perturbative expansion, we first add and subtract our approximate form of the integral about $T_c$ ($\int \tanh(x)/x$), as below.
\begin{align*}
\frac{1}{g\nu}& = \int_0^{\omega_D/2T}dx\ \Big[\frac{\tanh(x^2+\kappa^2)^{1/2}}{(x^2+\kappa^2)^{1/2}} -\frac{\tanh x}{x} \Big]\\&\hspace{3.35cm} +\int_0^{\omega_D/2T}\underbrace{dx\ \frac{\tanh x}{x}}_{\approx \ln(\omega_D/2T) }
\end{align*}
with $x=\xi/2T_c$ and $\kappa = \Delta/2T$. We then apply our previous approximation for the added integral, and expand with $\ln(\Delta/2T) = \ln(\Delta/2T_c) + \delta T/ T_c$ to arrive at the integral below:
\begin{align*}
\frac{1}{g\nu}& \approx \int_0^{\omega_D/2T}dx\ \Big[\frac{\tanh(x^2+\kappa^2)^{1/2}}{(x^2+\kappa^2)^{1/2}} -\frac{\tanh x}{x} \Big]\\&\hspace{4.85cm} +\frac{1}{g\nu} + \frac{\delta T}{T_c}\\
\\
-\frac{\delta T}{T_c}&=\int_0^{\omega_D/2T}dx\ \Big[\frac{\tanh(x^2+\kappa^2)^{1/2}}{(x^2+\kappa^2)^{1/2}} -\frac{\tanh x}{x} \Big]
\end{align*}
Which leaves us only to consider the remaining integral on the righthand side of the equality.

This integral can be treated approximately by splitting it into two effective parts. The first region, where $x$ ranges from zero to one, allows us to Taylor expand the hyperbolic tangent, resulting in a contribution of $C \times \kappa^2$. In the second region, with $x$ ranging from one to the upper limit of the integral ($\omega_D/2T$), the hyperbolic tangent may be approximated as one (similar to our previous approximation), which also results in a contribution of the form $C \times \kappa^2$. The sum of these two contributions then gives us the approximate equality below:
$$
-\frac{\delta T}{T_c}\approx C\times \kappa^2
$$
$$
\Rightarrow\quad \  \Delta \approx C\times \big(T_c(T_c-T)\big)^{1/2}\quad\quad
$$

where, again, $C$ is some numerical constant.
\section{Landau-Ginzberg Fluctuations}
\small
\begin{align*}
\chi_{n,\mathbf{q}}^c &= -\frac{k_BT}{V}\sum_{m,\mathbf{p}}G_{0}(\mathbf{p},i\omega_m)G_0(-\mathbf{p}+\mathbf{q},-i\omega_m+i\omega_n)\\
&=\frac{1}{V}\sum_p\frac{1-n_F(\xi_{p})-n_F(\xi_{p+q})}{i\omega_n-\xi_p-\xi_{-p+q}}
\end{align*}\normalsize
Now, converting the sum of momentum vectors into an integral (with the appropriate coefficients of $2\pi$), we have the following form for the zero
mode:
$$
\chi_{0,\mathbf{q}}^c = -\int \frac{d^d p}{(2\pi)^d}\frac{1-n_F(\xi_{p})-n_F(\xi_{p+q})}{-\xi_p-\xi_{-p+q}}
$$
The integral over all momenta here makes it transparent that we may make the 
transformation $p\rightarrow p-q/2$. Implementing this transformation, and noting that $\xi_p=\xi_{-p}$, we arrive at the form below.
$$
\chi_{0,\mathbf{q}}^c = \int \frac{d^d p}{(2\pi)^d}\frac{1-n_F(\xi_{p-q/2})-n_F(\xi_{p+q/2})}{\xi_{p-q/2}+\xi_{p+q/2}}
$$
Now, in the regions near mean field, we expect $q$ to be small. So, noting that the free particle dispersion, expanded about some point $k$ is the following:
$$
\epsilon_{k+q}=\frac{\hbar^2}{2m}\big(k^2+2k\cdot q + q^2 \big)
$$
We may approximate $\xi_p=\epsilon_k -\mu$ as $\xi_{p+q}\approx \xi_p + \frac{p\cdot q}{m}$, if we assume $q$ is small such that $q^2$ is negligible (and set $\hbar=1$). Applying this linear approximation, we are left with the following integral:
$$
\chi_{0,\mathbf{q}}^c = \int \frac{d^d p}{(2\pi)^d}\frac{1-n_F(\xi_{p})-\partial_{\xi}^2n_F(\xi_p)\frac{(q\cdot p)^2}{4m^2}}{2\xi_{p}}
$$
Now, we seek fluctuations from the mean field (where $q=0$). So noting, the form of the correlation function at $\omega_n=0$ and $q=0$, we have
$$
\chi_{0,\mathbf{q}}^c = \chi_{0,0}^c-\int \frac{d^d p}{(2\pi)^d}\frac{1-n_F(\xi_{p})-\partial_{\xi}^2n_F(\xi_p)\frac{(q\cdot p)^2}{4m^2}}{2\xi_{p}}
$$
Making the approximation that $p^2/2m\approx \mu$ at $\xi$ comparable to thermal energy $k_B$, and then converting the into one over energy, we have the following:
$$
\chi_{0,\mathbf{q}}^c = \chi_{0,0}^c-\frac{\nu \mu q^2}{12 m}\int \frac{d\epsilon}\epsilon^{-1} \partial^2_{\epsilon}n_F(\epsilon)
$$
And use the relation
$$
\int d\epsilon \epsilon^{-1}\partial_{\epsilon}^2 n_F(\epsilon) =\frac{7}{2\pi^2}\upsilon(s)\beta^{2}
$$
where $\upsilon(x)=\sum_{n=1}^{\infty}n^x$, to arrive at the final, quadratic form, below.
$$
\chi_{0,\mathbf{q}}^c = \chi_{0,0}^c-\frac{1}{24}\frac{7}{2\pi^2}\upsilon(s)\nu v_F^2\beta^{2}q^2
$$
\end{multicols}

\end{document}
